\documentclass[a4paper,oneside]{memoir}


\usepackage[T2A]{fontenc} % Поддержка русских букв
\usepackage[utf8]{inputenc} % Кодировка utf8
\usepackage[english,russian]{babel}

\usepackage{lipsum}
\usepackage{graphicx} % Required for including pictures

\usepackage{hyperref}
\usepackage{indentfirst}
\usepackage{amsmath}
\usepackage{amssymb}
\usepackage{amsfonts}
\usepackage{float} 
\usepackage{wrapfig}

%\linespread{1.5} % Line spacing

\title{Курс: "Комбинаторика для начинающих".
	
Неделя 2. Контрольная работа.
	 
Основные комбинаторные величины и их свойства.}

\newtheorem{task}{Задание}
\newtheorem{solution}{Решение}

 
\author{Александров Алексей, ИУ8-g4}

\date{2020г.}

\begin{document}
	
\maketitle

\begin{task}
	Есть 10 различных марок шоколада. Количество способов подарить трём девочкам по шоколадке равно?
\end{task}

\textbf{Ответ:} $ 10^{3} $

\begin{solution}
Каждая девочка может получить шоколадку любой марки. Поскольку все девочки и все марки различны, то количество способов подарить шоколадки девочкам в точности равно количеству размещений с повторениями $ \overline{A}_{10}^3 = 10^3 $
\end{solution}

\hrulefill

\begin{task}
Сколькими способами можно составить расписание авиарейсов на завтра, если всего имеется 7 рейсов, а в день осуществляется от двух до четырех перелётов?

\end{task}

\textbf{Ответ:} $ 1092 $

\begin{solution}
	Если в день осуществляется только два рейса, то первый можно выбрать 7 способами, а второй уже только 6 способами. То есть расписание из двух рейсов можно составить $ A_7^2 = 7\cdot 6 = 42 $ способами. Аналогично, если в день состоится 3 рейса, то расписание можно составить $ A_7^3 = 7\cdot 6\cdot 5 = 210 $ способами, а для 4 рейсов в день -- $ A_7^4 = 7\cdot 6\cdot 5\cdot 4 = 840 $. Всего получаем $ 42+210+840=1092 $ способа составить расписание.
\end{solution}

\hrulefill

\begin{task}
Сколькими способами можно вручить призы в 5 различных номинациях (в каждой номинации только один приз), если в соревновании участвуют 10 человек?
\end{task}

\textbf{Ответ:} $ 10^5 $

\begin{solution}
Приз в любой из 5 номинаций может достаться любому из 10 участников соревнований, то есть количество способов выдать призы в точности равно количеству размещений с повторениями $ \overline{A}_{10}^5 = 10^5 = 100000 $.
\end{solution}

\hrulefill

\begin{task}
Сколькими способами можно вручить призы в 5 различных номинациях (в каждой номинации только один приз), если в соревновании участвуют 10 человек, при условии, что каждый участник может получить не более одного приза?
\end{task}

\textbf{Ответ:} $ A_{10}^5 = 30240 $

\begin{solution}
Приз в первой номинации может достаться любому из 10 участников соревнований, во второй -- любому из 9 участников (получивший первый приз другие призы получать не может), в третьей -- любому из 8 участников, и т.д. То есть количество способов выдать призы в точности равно количеству размещений без повторений $ A_{10}^5 = 10\cdot 9\cdot 8\cdot 7\cdot 6 = 30240 $.
\end{solution}

\hrulefill

\begin{task}
	В семье семеро детей: старший -- мальчик, дальше -- девочка, девочка, мальчик, мальчик, мальчик и младшая -- девочка. Сколькими способами родители могут выбрать имена, если они выбирают из 10 мужских и 13 женских имен и хотят, чтобы имена не повторялись?
\end{task}

\textbf{Ответ:} $ 8648640 $

\begin{solution}
	Количество способов выбрать имена сыновьям равно количеству размещений без повторений $ A_{10}^4=\dfrac{10!}{6!} $, аналогично, количество способов выбрать имена дочерям равно $ A_{13}^3 = \dfrac{13!}{10!} $. Общее количество способов выбрать имена детям равно $ \dfrac{10!}{6!}\cdot\dfrac{13!}{10!} = \dfrac{13!}{6!} = 7\cdot\ldots\cdot 13 = 8648640 $.
\end{solution}

\hrulefill

\begin{task}
	Жених и невеста выбирают трехъярусный свадебный торт. На выбор имеются 5 типов ярусов (бисквитный, йогуртовый, чизкейк и т.д.). Сколько различных тортов может предложить кондитер, если бисквитных ярусов может быть не больше двух, а ярусов любого другого типа не больше одного?
\end{task}

\textbf{Ответ:} $ 72 $

\begin{solution}
Нужно рассмотреть 3 случая: в торте нет бисквитных ярусов, ровно один бисквитный ярус и ровно два бисквитных яруса.

В первом случае (бисквитный ярус использовать нельзя) количество способов испечь торт равно количеству размещений без повторений $ A_4^3 = 24 $.

Во втором случае (ровно один бисквитный ярус) нужно сначала разместить для бисквитный ярус, для это всего 3 варианта, а затем выбрать оставшиеся два яруса из 4 типов и определить для них места, то есть $ A_4^2 = 12 $ способов. Всего получаем $ 3\cdot 12=36 $ тортов.

В третьем случае (ровно два бисквитных яруса) нужно выбрать место для "небисвкитного" яруса на любом из 33 уровней, а затем выбрать тип для этого яруса из оставшихся 44 типов $ A_4^1 = 4 $ способами. Всего получаем $ 3\cdot 4 = 12 $ тортов.

Суммируя результаты для всех трех случаев, получаем $ 24+36+12 = 72 $ различных торта.
\end{solution}

\hrulefill

\begin{task}
В университете десятибальная система оценок: 1-2 -- "неудовлетворительно", 3-4 -- "удовлетворительно", 5-7 -- "хорошо" и 8-10 -- "отлично". Сколькими способами можно поставить оценки 5 студентам, если известно, что экзамен сдали все (т.е. нет неудовлетворительных оценок)?
\end{task}

\textbf{Ответ:} $ 32768 $

\begin{solution}
 Каждый студент может получить любую из 8 оценок от 3 до 10. То есть общее число способов поставить удовлетворительные и более высокие оценки студентам равно количеству размещений с повторениями $ \overline{A}_{8}^5 = 8^5 = 32768 $.
\end{solution}

\hrulefill

\begin{task}
	Группа из 4 студентов пришла в столовую. Сколькими способами они могут занять очередь друг за другом, если Маша и Таня хотят стоять рядом, а Коля не хочет быть последним?
\end{task}

\textbf{Ответ:} $  $

\begin{solution}
	Приведём здесь решение перебором. Другое решение, которое годится и в общем случае написано в решении к следующей задаче.
	
	Пара Маша-Таня занимает либо 1-2, либо 2-3, либо 3-4 места. Это 6 вариантов.
	
	Если пара Маша-Таня занимает 1-2 или 2-3 места, то Коля занимает 3 и 1 место соответственно, а оставшийся студент - 4 место.
	
	Если пара Маша-Таня занимает 3-4 место, то Коля занимает 1 или 2 место. То есть в этом случае вариантов будет 2.
	
	Итого: $ 2+2+2 \cdot 2=8 $ вариантов.
\end{solution}

\begin{task}
	Группа из 8 студентов пришла в столовую. Сколькими способами они могут занять очередь друг за другом, если Маша и Таня хотят стоять рядом, а Коля не хочет быть последним?
\end{task}

\textbf{Ответ:} $ 8640 $

\begin{solution}
	Так как Маша и Таня хотят стоять рядом, то можно считать, что они занимают одно место вдвоём. Соответственно, мест становится 7, но надо учесть, что девочки могут между собой поменяться местами (Маша-Таня и Таня-Маша). Поэтому количество всех очередей надо будет удвоить.
	
	Коля не хочет быть последним, поэтому для него есть 6 возможных мест в очереди. Остальные могут занять места $ 6 \cdot 5 \dots 1= 6! $ способами.
	
	В итоге получаем, что количество способов занять очередь равно $ 2\cdot 6 \cdot 6! = 8640 $.
\end{solution}

\end{document}



