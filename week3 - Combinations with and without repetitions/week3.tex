\documentclass[a4paper,oneside]{memoir}


\usepackage[T2A]{fontenc} % Поддержка русских букв
\usepackage[utf8]{inputenc} % Кодировка utf8
\usepackage[english,russian]{babel}

\usepackage{lipsum}
\usepackage{graphicx} % Required for including pictures

\usepackage{hyperref}
\usepackage{indentfirst}
\usepackage{amsmath}
\usepackage{amssymb}
\usepackage{amsfonts}
\usepackage{float} 
\usepackage{wrapfig}

%\linespread{1.5} % Line spacing

\title{Курс: "Комбинаторика для начинающих".
	
Неделя 3. Контрольная работа.
	 
Сочетания с повторениями и без.}

\newtheorem{task}{Задание}
\newtheorem{solution}{Решение}

 
\author{Александров Алексей, ИУ8-g4}

\date{2020г.}

\begin{document}
	
\maketitle

\begin{task}
В одном маленьком королевстве выбирают царя, его советника и пятерых стражников из 30 кандидатов. Сколькими способами это можно сделать?
\end{task}

\textbf{Ответ:} $ 30 \cdot 29 \cdot C_{28}^{5} $

\begin{solution}
Должности царя, советника и стражника различны, поэтому важно кому из кандидатов какая должность достанется. Если царя можно выбрать 30 способами, то стать его советником сможет уже один из 29 претендентов. После чего останется выбрать 5 стражников из оставшихся 28 человек. Поскольку должности стражников между собой никак не отличаются, то речь идет о сочетании без повторений $  C_{28}^5 $.

В итоге получаем, что выбрать людей на искомые должности можно $ 30\cdot 29\cdot C_{28}^5 $ способами.
\end{solution}

\hrulefill

\begin{task}
	У королевы есть 12 одинаковых зеркал. Сколькими способами их можно повесить в 8 разных залах замка так, чтобы в каждом зале было хотя бы одно зеркало?
\end{task}

\textbf{Ответ:} $ \overline{C}_{8}^4 = 330 $

\begin{solution}
	Попробуем понять, что мы из чего выбираем. Так как в каждом зале висит хотя бы одно зеркало, то можно считать, что мы уже повесили в каждом зале по зеркалу, и остались 4 зеркала, которые надо развесить произвольным образом по 8 залам (тогда в зале может висеть от одного зеркала до пяти). Если занумеровать все залы, то получится, что нам нужно выбрать четыре числа (некоторые из них могут повторяться) из восьми возможных, причем порядок этих чисел не важен (сколько раз число встретилось в выбранном наборе, столько "дополнительных" зеркал в соответствующем зале). Это в точности количество сочетаний с повторениями $ \overline{C}_{8}^4 = C_{11}^{4} =330 $.
\end{solution}

\hrulefill
\begin{task}
	В летнем лагере 10 отрядов. Каждый день дежурят по 3 отряда. Сколькими способами можно выбрать дежурные отряды?
\end{task}

\textbf{Ответ:} $ C_{10}^3 $

\begin{solution}
	Для каждого дня нам надо выбрать 3 отряда из 10. Последовательность выбора не важна, следовательно, это сочетание. Отряды не повторяются, поэтому это сочетание без повторений. Количество способов выбрать 3-сочетание без повторений из 10 элементов равно $ C_{10}^3 = \frac{10!}{3!\cdot7!} $.
\end{solution}

\hrulefill
\begin{task}
	В палитре художника имеется 10 разных красок. Сколько разных оттенков он может получить, смешивая 3 краски (краски могут быть одинаковыми, получаемый оттенок не зависит от порядка смешивания красок)?
\end{task}

\textbf{Ответ:} $ \overline{C}_{10}^3 $

\begin{solution}
	Мы выбираем 3 краски из 10. Порядок роли не играет, следовательно, мы выбираем сочетание. Краски могут быть одинаковыми, следовательно, это сочетание с повторениями. Итоговый ответ: $ \overline{C}_{10}^3 = \frac{12!}{3! \cdot 9!} $.
\end{solution}

\hrulefill
\begin{task}
	Команда «Турин» состоит из 10 баскетболистов, а команда «Ювента» -- из 12. Сколькими способами команды могут сформировать стартовый состав (в баскетболе на поле играют 5 человек)?
\end{task}

\textbf{Ответ:} $ C_{10}^5 \cdot C_{12}^5 =199584 $

\begin{solution}
	Для каждой команды мы выбираем 5 баскетболистов из 10 или 12. Порядок выбора не важен, баскетболисты не повторяются, следовательно, это сочетание без повторений. Для «Турина» количество способов выбора равно $ C_{10}^5 $, для «Ювенты» -- $ C_{12}^5C $. По правилу умножения получаем $ C_{10}^5 \cdot C_{12}^5 =199584 $.
\end{solution}

\hrulefill
\begin{task}
	Сколькими способами в течение 5 дней можно выбирать на дежурство по 4 ученика из класса в 20 человек так, чтобы каждый день состав дежурных был разным?
\end{task}

\textbf{Ответ:} $ A_{C_{20}^4}^5 $

\begin{solution}
	Каждый день мы выбираем 4 учеников из 20. Так как порядок выбора не важен, выбор без повторений, то это 4-сочетание без повторений из 20 элементов. Количество таких сочетаний равно $ C_{20}^4 $. Общее количество способов -- это выбор дежурных на 5 дней (порядок выбора важен, дежурные не повторяются) -- 5-размещение без повторений из $ C_{20}^4 $ элементов. Значит, искомое количество равно $ A_{C_{20}^4}^5 = \frac{(C_{20}^4)!}{(C_{20}^4 - 5)!} = C_{20}^4\cdot (C_{20}^4 - 1)\cdot (C_{20}^4 - 2)\cdot (C_{20}^4 - 3)\cdot (C_{20}^4 - 4) $.
\end{solution}

\hrulefill

\begin{task}
	Имеется 10 разных чашек и 5 разных ложек. Количество способов разложить ложки по чашкам (в одну чашку влезает сколько угодно ложек) является по своей сути чем?
\end{task}

\textbf{Ответ:} $ 10^5 $

\begin{solution}
	Для каждой ложки мы выбираем чашку. Ложки различные, значит порядок раскладывания важен. В одну чашку можно положить несколько ложек, следовательно, мы размещаем ложки по чашкам с повторением.
	
	Для наглядности посчитаем количество этих способов. Для каждой ложки есть 10 вариантов чашек, поэтому общее количество вариантов по правилу умножение равно $ 10^5 $.
\end{solution}

\hrulefill
\begin{task}
	В первый класс набрали 75 учеников. Сколькими способами можно распределить их по трем классам по 25 человек в каждом?
\end{task}

\textbf{Ответ:} $ C_{75}^{25} \cdot C_{50}^{25} $

\begin{solution}
	Можно организовать распределение следующим образом: сначала выбрать 25 учеников из 75 в первый класс, а потом из оставшихся 50 выбрать 25 во второй класс, тогда последние 25 учеников автоматически попадут в третий класс. В каждом выборе порядок не важен и повторов нет, поэтому это сочетания без повторений: $ C_{75}^{25} $ , $ C_{50}^{25} $. По правилу умножения получаем, что искомое количество способов равно $ C_{75}^{25} \cdot C_{50}^{25} $.
\end{solution}

\end{document}



