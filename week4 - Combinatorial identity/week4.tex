\documentclass[a4paper,oneside]{memoir}


\usepackage[T2A]{fontenc} % Поддержка русских букв
\usepackage[utf8]{inputenc} % Кодировка utf8
\usepackage[english,russian]{babel}

\usepackage{lipsum}
\usepackage{graphicx} % Required for including pictures

\usepackage{hyperref}
\usepackage{indentfirst}
\usepackage{amsmath}
\usepackage{amssymb}
\usepackage{amsfonts}
\usepackage{float} 
\usepackage{wrapfig}

%\linespread{1.5} % Line spacing

\title{Курс: "Комбинаторика для начинающих".
	
Неделя 4. Контрольная работа.
	 
Комбинаторные тождества.}

\newtheorem{task}{Задание}
\newtheorem{solution}{Решение}

 
\author{Александров Алексей, ИУ8-g4}

\date{2020г.}

\begin{document}
	
\maketitle

\begin{task}
	Шестая строчка треугольника Паскаля выглядит следующим образом:
\end{task}

\textbf{Ответ:}  1 6 15 20 15 6 1 

\begin{solution}
По определению, шестая строчка треугольника Паскаля получается из пятой 1 5 10 10 5 1 суммированием чисел стоящих слева сверху и справа сверху. Получаем следующую строчку: 1 6 15 20 15 6 1.
\end{solution}

\hrulefill

\begin{task}
	На дереве висит 10 разных яблок. Сколькими способами можно сорвать нечётное количество яблок?
\end{task}

\textbf{Ответ:} $  $

\begin{solution}
	Из задачи «Наборы из чётного числа символов» мы знаем, что чётное количество яблок можно сорвать $ 2^{10-1} = 2^9 = 512 $ способами. Так как общее количество способов сорвать яблоки равно $ 2^{10} $, то нечётное количество яблок можно сорвать также $ 512 $ способами.
\end{solution}

\hrulefill

\begin{task}
	Сумма $ C_{10}^1+C_{10}^2+\ldots+C_{10}^{10} = \sum\limits_{i=1}^{10}C_{10}^i $ равна
\end{task}

\textbf{Ответ:} $ 10^{10} - 1 = 1023 $

\begin{solution}
	Мы знаем, что $ C_{n}^0 + C_n^1+ \ldots +C_n^n = 2^n $. Подставляя n=10, получаем, что $ C_{10}^0 + C_{10}^1+C_{10}^2+\ldots+C_{10}^{10} = 2^{10}=1024 $. Наша сумма получается из данной вычитанием $ C_{10}^0 = 1 $. Следовательно, ответ равен 1023.
\end{solution}

\hrulefill

\begin{task}
	 Коэффициент при $ x^7 $ в разложении $ (1+x)^{11} $ равен:
\end{task}

\textbf{Ответ:} $ C_{11}^7 $

\begin{solution}
	По формуле бинома Ньютона коэффициент при $ x^7 $ равен $ C_{11}^7 = C_{11}^4 $ .
\end{solution}

\hrulefill

\begin{task}
	В наборе из 12 сосудов имеется 5 неразличимых стаканов и 7 различных чашек. Сколькими способами можно выбрать 6 сосудов из 12?
\end{task}

\textbf{Ответ:} $  $

\begin{solution}
	Для каждого фиксированного k существует только один способ выбрать k неразличимых стаканов. Отсюда искомое количество способов равно количеству способов выбрать от 1 до 6 чашек. Искомая сумма равна $ C_{7}^1+ \ldots + C_{7}^6 =C_{7}^0+ C_{7}^1+ \ldots + C_{7}^6 + C_7^7 - (C_7^0+C_7^7) = 2^{7} - 2 =128-2=12 $6.
\end{solution}

\hrulefill

\begin{task}
	Сумма $ C_{n+m-1}^m+C_{n+m-2}^m+\ldots + C_{m}^m $ $\forall m \ge 1 $, $ n \ge 1 $ равна:
\end{task}

\textbf{Ответ:} $ C_{n+m}^{m+1} = C_{n+m}^{n-1} $

\begin{solution}
	Эта сумма в точности равна сумме чисел в треугольнике Паскаля, расположенных на одной диагонали, начиная с числа $ C_{n + m - 1}^m $ и выше. Эта задача разобрана на видео, и ответ -- число, стоящее под $ C_{n+m-1}^m $ справа, то есть $ C_{n+m}^{m+1} = C_{n+m}^{n-1} $.
\end{solution}

\hrulefill

\begin{task}
	Отметьте тождества, выполненные $ \forall n \ge k \ge 0$ .
\end{task}

\begin{solution}
	
	$ 2^n = \sum\limits_{i=0}^{n}C_{n}^i $ -- верно;
	
	
	$ 0 = C_n^0 - C_n^1 + \ldots + (-1)^n C_n^n0 $ -- неверно для n=0;
	
	
	$ C_{n-k}^k = C_{n-k}^{n-k} $ -- неверно для n=3, k=1;
	
	
	$ C_{n}^k = C_{n-1}^k + C_{n-1}^{k+1} $ -- неверно для n=4, k=1 $ (4=C_4^1 \neq C_3^1 + C_3^2 = 3 + 3 =6 $.
\end{solution}


\end{document}