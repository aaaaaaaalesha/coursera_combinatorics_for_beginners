\documentclass[a4paper,oneside]{memoir}


\usepackage[T2A]{fontenc} % Поддержка русских букв
\usepackage[utf8]{inputenc} % Кодировка utf8
\usepackage[english,russian]{babel}

\usepackage{lipsum}
\usepackage{graphicx} % Required for including pictures

\usepackage{hyperref}
\usepackage{indentfirst}
\usepackage{amsmath}
\usepackage{amssymb}
\usepackage{amsfonts}
\usepackage{float} 
\usepackage{wrapfig}

%\linespread{1.5} % Line spacing

\title{Курс: "Комбинаторика для начинающих".
	
Неделя 5. Контрольная работа.
	 
Полиномиальные коэффициенты.}

\newtheorem{task}{Задание}
\newtheorem{solution}{Решение}

 
\author{Александров Алексей, ИУ8-g4}

\date{2020г.}

\begin{document}
	
\maketitle

\begin{task}
Количество различных слов, получаемых перестановкой букв в слове \textbf{папарацци} равно
\end{task}

\textbf{Ответ:} $ P(3,2,2,1,1) = 15120 $

\begin{solution}
В слове папарацци 2 раза повторяются буквы \textbf{п}, \textbf{ц}, 3 раза \textbf{а}, по одному разу \textbf{р} и \textbf{и}. То есть из набора букв  \textit{(а, п, ц, р, и)} мы составляем слова в которых указанные буквы встречаются нужное количество раз. Это в точности полиномиальный коэффициент $ P(3,2,2,1,1) = \dfrac{9!}{3!\cdot 2! \cdot 2! \cdot 1! \cdot 1!} = 15120 $.
\end{solution}

\hrulefill

\begin{task}
	Имеется 18 различных шаров и 4 различных ящика. Сколькими способами можно в первые два ящика положить по 5 шаров, а в оставшиеся два -- по 4 шара (Отметьте все подходящие варианты)?
\end{task}

\textbf{Ответ:} $ P(5,4,4,5) $

\begin{solution}
	Занумеруем ящики \{1,2,3,4\} . Тогда каждому способу разложить шары по ящикам можно поставить в соответствие последовательность из \{1,2,3,4\} длины 18, причём 1 и 2 встречается по 5 раз, а 3 и 4 -- по 4. Отсюда искомое количество способов равно $ P(5,5,4,4) =\dfrac{18!}{5!\cdot 5! \cdot 4! \cdot 4!} = P(5,4,4,5) $).
\end{solution}

\hrulefill

\begin{task}
	У продавца антиквариата имеется 12 разных монет. Четверо нумизматов (Андрей, Борис, Виктор и Геннадий) купили эти монеты: Андрей и Борис по 4 монеты, а Виктор и Геннадий -- по 2 монеты. Сколькими способами они могли осуществить свои покупки?
\end{task}

\textbf{Ответ:} $ P(4,4,2,2) = 207900 $

\begin{solution}
	Данная задача аналогична предыдущим: можно занумеровать нумизматов или считать, что мы собираем "слово" из четырёх букв А, четырёх букв Б, двух В и двух Г. Ответ: $ P(4,4,2,2) = \dfrac{12!}{4!\cdot 4! \cdot 2! \cdot 2!} = 207900 $.
\end{solution}

\hrulefill

\begin{task}
	У продавца антиквариата имеется 12 разных монет. Четверо нумизматов купили эти монеты: какие-то двое по 4 монеты, а оставшиеся двое -- по 2 монеты. Сколькими способами они могли осуществить свои покупки?
\end{task}

\textbf{Ответ:} $  $

\begin{solution}
	В чём отличие данной задачи от предыдущей? Теперь мы дополнительно выбираем, кому достанется 4 монеты, а кому 2. Аналогично задаче про цветки и девочек, выбираем двух нумизматов, котором достанется по 4 монеты (это можно сделать $ C_4^2 $ способами). А затем, как в предыдущей задаче раздаём им монеты $ P(4,4,2,2) $ способами. Итого, получаем: $ C_4^2 \cdot P(4,4,2,2) = 6 \cdot 207900 = 1247400 $.
\end{solution}

\hrulefill

\begin{task}
	Полиномиальный коэффициент $ P(5,4,3,2,1) $ равен:
\end{task}

\textbf{Ответ:} $ \dfrac{15!}{5!\cdot 4! \cdot 3! \cdot 2! \cdot 1!} $

\begin{solution}
	По полиномиальной формуле, $ P(5,4,3,2,1) = \dfrac{15!}{5!\cdot 4! \cdot 3! \cdot 2! \cdot 1!} $.
	
	
	$ 	C_{15}^{10} \cdot C_{10}^6 \cdot C_{6}^3 \cdot C_3^2 = \dfrac{15!}{10! \cdot 5!} \cdot \dfrac{10!}{6! \cdot 4!} \cdot \dfrac{6!}{3! \cdot 3!} \cdot \dfrac{3!}{2!\cdot 1!} = \dfrac{15!}{5! \cdot 4! \cdot 3! \cdot 2! \cdot 1!} $.
	
	$ C_{15}^{5} \cdot C_{10}^4 \cdot C_{6}^3 \cdot C_3^2 = \dfrac{15!}{5! \cdot 10!} \cdot \dfrac{10!}{4! \cdot 6!} \cdot \dfrac{6!}{3! \cdot 3!} \cdot \dfrac{3!}{2!\cdot 1!} = \dfrac{15!}{5! \cdot 4! \cdot 3! \cdot 2! \cdot 1!} $.
\end{solution}

\hrulefill

\begin{task}
	Чему равна сумма, приведённая ниже?
	
	$ \sum\limits_{(n_1,n_2,n_3), n_1+n_2+n_3=15, n_i \geq 0, n_i \in \mathbb{Z}} (-1)^{n_3} P(n_1,n_2,n_3) $ 
	
\end{task}

\textbf{Ответ:} $ 1 $

\begin{solution}
$ \sum\limits_{(n_1,n_2,n_3), n_1+n_2+n_3=15, n_i \geq 0, n_i \in \mathbb{Z}} P(n_1,n_2,n_3) \cdot 1^{n_1} \cdot 1^{n_2} \cdot (-1)^{n_3} = $

\begin{center}
	$= (1+1-1)^{15} = 1 $.
\end{center}
\end{solution}

\hrulefill

\begin{task}
	В чемпионате Европы по футболу участвуют 24 команды. Золотые медали получает команда победитель, серебряные -- команда, проигравшая в финале, бронзовые -- две команды, которые проиграли в полуфинале. Сколькими способами могут распределиться медали между командами? (Отметьте все правильные варианты)
\end{task}

\textbf{Ответ:} $ P(20,1,1,2) $

\begin{solution}
	Из 24 команд надо выбрать 1 победителя, 1 финалиста, 2 полуфиналиста и 20 команд, не попавших в полуфинал. Количество способов -- полиномиальный коэффициент $ P(20,1,1,2) =  \frac{24!}{20! \cdot 1! \cdot 1! \cdot 2!} = \frac{24 \cdot 23 \cdot 22 \cdot 21}{2!} $.
\end{solution}

\hrulefill

\begin{task}
	Коэффициент при $ x^{10} $ в разложении $ (1+x^2+x^3)^6 $ равен:
\end{task}

\textbf{Ответ:} $ P(1,5,0) + P(2,2,2) = 96 $

\begin{solution}
	Чтобы получить $ x^{10} $ надо сколько-то раз взять $ x^2 $, а сколько-то $ x^3 $ (а из оставшихся скобок взять 1). Какие могут быть варианты? Надо взять $ x^3 $ чётное количество раз, то есть 0 или 2. Отсюда получаем два способа получить $ x^{10} $. Первый: из 5 скобок взять $ x^2 $, а из оставшейся -- 11. Таких способов всего $ P(1,5,0) $. Второй способ: из двух скобок взять 11, из двух -- $ x^2 $, и из оставшихся двух -- $ x^3 $. Количеств таких способов равно $ P(2,2,2) $. Итого получаем сумму $ P(1,5,0) + P(2,2,2) $.
\end{solution}



\end{document}