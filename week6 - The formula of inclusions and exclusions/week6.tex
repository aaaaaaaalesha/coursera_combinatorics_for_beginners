\documentclass[a4paper,oneside]{memoir}


\usepackage[T2A]{fontenc} % Поддержка русских букв
\usepackage[utf8]{inputenc} % Кодировка utf8
\usepackage[english,russian]{babel}

\usepackage{lipsum}
\usepackage{graphicx} % Required for including pictures

\usepackage{hyperref}
\usepackage{indentfirst}
\usepackage{amsmath}
\usepackage{amssymb}
\usepackage{amsfonts}
\usepackage{float} 
\usepackage{wrapfig}

%\linespread{1.5} % Line spacing

\title{Курс: "Комбинаторика для начинающих".
	
Неделя 6. Контрольная работа.
	 
Формула включений-исключений.}

\newtheorem{task}{Задание}
\newtheorem{solution}{Решение}

 
\author{Александров Алексей, ИУ8-g4}

\date{2020г.}

\begin{document}
	
\maketitle

\begin{task}
	Переплётчик должен переплести 5 различных книг в красный, зелёный или коричневый переплёты. Сколькими способами он может это сделать, если в каждый цвет должна быть переплетена хотя бы одна книга? Все книги различны.
\end{task}

\textbf{Ответ:} $ 150 $

\begin{solution}
См. решение задачи 2.
\end{solution}

\hrulefill

\begin{task}
Сколькими способами можно расселить 5 туристов по 3 домикам, чтобы ни один домик не оказался пустым? Все туристы и домики различны. Способы расселения, отличающиеся только перестановкой туристов, заселённых в один домик, считаются одинаковыми.
\end{task}

\textbf{Ответ:} $ 150 $

\begin{solution}
В качестве множеств $ A_i $ (или свойства $ \alpha_i $) рассмотрим множества расселений туристов по домикам, при которых i-ый домик является пустым. Тогда по формуле включений и исключений мы можем найти количество расселений, при которых ни одно из свойства $ \alpha_i $ не выполнено. То есть ни один домик не является пустым, что и требуется найти. Таким образом, искомое количество расселений n находится по формуле:

$ n = |X| - |A_1| - |A_2| - |A_3| + |A_1 \cap A_2| + |A_1 \cap A_3| + |A_2 \cap A_3| - |A_1 \cap A_2 \cap A_3| $


$ |X| $ -- это общее количество расселений, то есть $ 3^5 $. $ |A_i| $ -- это количество способов расселить туристов по не более чем двум домикам, т.е. $ 2^5 $, $ |A_i \cap A_j| $-- это количество способов расселить туристов в оставшийся домик (не i и не j), то есть $ 1^5 $, а $ |A_1 \cap A_2 \cap A_3| = 0 $, так как домиков для расселения не осталось.


Итого получаем: $ n = 3^5 - 3\cdot 2^5 + 3 \cdot 1^5 - 0 = 243 - 3\cdot 32 + 3 = 150 $.
\end{solution}

\hrulefill

\begin{task}
	Дана таблица размером $ 2 \times 5 $. В левом верхнем углу записано число 11. Сколькими способами таблицу можно дополнить числами \{1,2,3,4,5\} так, чтобы выполнялись оба следующих условия:
	
	1) в каждой строчке присутствовало каждое из чисел от 1 до 5
	
	2) в каждом столбце все числа были различны?
	
	(Пример такого заполнения: первая строчка: \textbf{1,2,5,4,3} вторая строчка: \textbf{3,5,2,1,4}.)
\end{task}

\textbf{Ответ:} $ 44 \cdot 4! = 1056 $

\begin{solution}
	Первую строчку можно заполнить 4! способами. Вторую строчку надо заполнить так, чтобы ни один элемент в нижней строке не находился в том же столбце. Поэтому это число равно числу беспорядков из 5 элементов, то есть 44 (см. следующую задачу). Итого: $ 44 \cdot 4! = 1056 $.
\end{solution}

\hrulefill
\begin{task}
	Число беспорядков в последовательности из 5 элементов равно:
\end{task}

\textbf{Ответ:} $ 5! \cdot (1 - \frac{1}{1!} + \frac{1}{2!} - \frac{1}{3!} + \frac{1}{4!} - \frac{1}{5!}) = 44 $

\begin{solution}
	 По формуле из лекции получаем: $ 5! (1 - \frac{1}{1!} + \frac{1}{2!} - \frac{1}{3!} + \frac{1}{4!} - \frac{1}{5!}) = 5! (\frac{1}{2} - \frac{1}{6} + \frac{1}{24} - \frac{1}{120}) = 60 - 20 + 5 - 1 = 44 $.
\end{solution}

\hrulefill
\begin{task}
	На загородную прогулку поехали 92 человека. Бутерброды с колбасой взяли 48 человек, с сыром — 38 человек, с сыром и колбасой — 28 человек. Сколько человек не взяли с собой бутерброды?
\end{task}

\textbf{Ответ:} $ 34 $

\begin{solution}
	По формуле включений и исключений получаем: $ 92-48-38+28 = 34 $.
\end{solution}

\hrulefill
\begin{task}
	Формула включений и исключений для трёх множеств A,B,CA,B,C выглядит следующим образом:
\end{task}

\textbf{Ответ:} $ |A \cup B \cup C| = |A| + |B| + |C| - |A \cap B| - |A \cap C| - |B \cap C| + |A \cap B \cap C| $

\begin{solution}
	
\end{solution}
 См. лекцию.
\end{document}