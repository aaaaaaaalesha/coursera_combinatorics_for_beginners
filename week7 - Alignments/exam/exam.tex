\documentclass[a4paper,oneside]{memoir}


\usepackage[T2A]{fontenc} % Поддержка русских букв
\usepackage[utf8]{inputenc} % Кодировка utf8
\usepackage[english,russian]{babel}

\usepackage{lipsum}
\usepackage{graphicx} % Required for including pictures

\usepackage{hyperref}
\usepackage{indentfirst}
\usepackage{amsmath}
\usepackage{amssymb}
\usepackage{amsfonts}
\usepackage{float} 
\usepackage{wrapfig}
\usepackage[warn]{mathtext}
\DeclareSymbolFont{T2Aletters}{T2A}{cmr}{m}{it}

%\linespread{1.5} % Line spacing

\title{Курс: "Комбинаторика для начинающих".
	
Экзамен.}

\newtheorem{task}{Задание}
\newtheorem{solution}{Решение}

 
\author{Александров Алексей, ИУ8-g4}

\date{2020г.}

\begin{document}
	
\maketitle

\begin{task}
Сколько четырехзначных чисел не имеют в своей десятичной записи одинаковых цифр?
\end{task}

\textbf{Ответ:} $ 4536 $

\begin{solution}
На первое место четырехзначного числа можно поставить любую цифру, кроме 0, т.е. всего 9 вариантов; на второе место -- любую цифру, кроме той, что стоит на первом месте (9 вариантов); на третье -- любую, кроме двух, стоящих на первом и втором месте (8 вариантов); аналогично, для последней цифры остается 7 вариантов. В итоге по правилу умножения получаем: $ 9\cdot 9\cdot 8\cdot 7 = 4536 $.
\end{solution}

\hrulefill

\begin{task}
	В ящике лежат 30 черных, 30 белых и 30 красных носков. Какое минимальное количество носков надо вытащить, чтобы среди них гарантированно было 2 носка одного цвета?
\end{task}

\textbf{Ответ:} $ 4 $

\begin{solution}
	В ящике находятся носки трех цветов. Если вытащить не меньше четырех носков, то по принципу Дирихле найдется хотя бы два носка одного цвета. Значит, минимальное количество равно 4.
\end{solution}

\hrulefill


\begin{task}
	В саду растёт яблоня, груша и лимонное дерево. На каждом дереве по 20 плодов (все плоды разные). Сколькими способами можно выбрать два яблока, одну грушу и один лимон?
\end{task}

\textbf{Ответ:} $ 76000 $

\begin{solution}
	Количество способов выбрать из 20 различных объектов 2 (порядок не важен) равно $ C_{20}^2 = 190 $, количество способов выбрать из 20 различных объектов 1 равно $ C_{20}^1 = 20 $. По правилу умножения получаем $ 190\cdot 20\cdot 20 = 76000 $.
\end{solution}

\hrulefill

\begin{task}
	В группе музыкантов трое гитаристов и 20 гитар. Сколькими способами музыканты могут выбрать себе по инструменту?
\end{task}

\textbf{Ответ:} $ A_{20}^3 = 6840 $

\begin{solution}
	Необходимо из 20 различных объектов выбрать 3, но поскольку все люди уникальные, то важно, какому человеку какой инструмент достанется, т.е. важен порядок выбора. Значит, искомое количество равно $ A_{20}^3 = 18\cdot 19\cdot 20 = 6840 $.
\end{solution}

\hrulefill

\begin{task}
	Коля загадал последовательность из чисел 1,2,3,4,5, причём ни одно из чисел не стоит на своём месте. Вася пытается угадать эту последовательность (последовательность считается угаданной, если Вася написал её на бумаге и показал Коле). Сколько вариантов надо перебрать Васе, чтобы гарантированно угадать последовательность?
\end{task}

\textbf{Ответ:} $ 44 $

\begin{solution}
	Необходимо посчитать количество беспорядков для 55 элементов (см. раздел 6 "Формула включений и исключений" задача "Количество беспорядков"). Искомое количество равно $ 5!\left(\frac{1}{0!}-\frac{1}{1!}+\frac{1}{2!}-\frac{1}{3!}+\frac{1}{4!}-\frac{1}{5!}\right) = \frac{5!}{2!}-\frac{5!}{3!}+\frac{5!}{4!}-\frac{5!}{5!} = 60-20+5-1=44 $.
\end{solution}

\hrulefill

\begin{task}
	 В институте работают 100 сотрудников, причём: английский знают 60 сотрудников; не знают немецкий 55 сотрудников; французский знают 25 сотрудников; 25 сотрудников не знают ни французский, ни английский; английский и немецкий знают 15 сотрудников; французский и немецкий знают 15 сотрудников; все три языка знают 5 сотрудников. Отметьте верные утверждения.
\end{task}

\textbf{Ответ:} $  $

\begin{solution}
	Пусть $ \alpha_a (\alpha_n, \alpha_f) $ -- свойство "знать английский (немецкий, французский) язык", тогда по формуле включений и исключений можно получить следующее.
	
$ 	N(\alpha_n') = N - N(\alpha_n) \Rightarrow N(\alpha_n) = N - N(\alpha_n') = 100-55=45. $ 

	
	$ N(\alpha_a', \alpha_f') = N - N(\alpha_a) - N(\alpha_f)+N(\alpha_a, \alpha_f) \Rightarrow $
	
	$ \Rightarrow N(\alpha_a, \alpha_f) = N(\alpha_a', \alpha_f') - N +N(\alpha_a) + N(\alpha_f) $
	
	$ =25-100+60+25=10 $.
	
	Не знает ни одного языка $ N(\alpha_a', \alpha_n', \alpha_f') = N - N(\alpha_a) - N(\alpha_n) - N(\alpha_f) + N(\alpha_a, \alpha_n) + N(\alpha_a, \alpha_f) + N(\alpha_n, \alpha_f) - N(\alpha_a, \alpha_n, \alpha_f)=100-60-45-25+15+10+15-5=5 $ человек.
	
	Знает хотя бы один язык $ N - N(\alpha_a', \alpha_n', \alpha_f') = 100-5=95 $ человек.
	
	Знает немецкий, французский, но не знает английский язык $ N(\alpha_a', \alpha_n, \alpha_f) = N(\alpha_n, \alpha_f) - N(\alpha_a, \alpha_n, \alpha_f)=15-5=10 $ человек.

	Знает французский, но не знает ни английского, ни немецкого языков $ N(\alpha_a', \alpha_n', \alpha_f) = N(\alpha_f) - N(\alpha_a, \alpha_f) - N(\alpha_n, \alpha_f)+N(\alpha_a, \alpha_n, \alpha_f)=25-10-15+5 = 5 $ человек.
\end{solution}

\hrulefill

\begin{task}
	Имеются слова "коала" и "лайка". Сколько выравниваний есть у этих двух слов?
\end{task}

\textbf{Ответ:} $ 252 $

\begin{solution}
	Количество выравниваний двух слов длины 55 каждое равно $ g(5,5)=C_{5+5}^5 = C_{10}^5 =\frac{10!}{5!\cdot 5!}= 252 $ (см. раздел 7 "Выравнивания" лекция "Следствие из теоремы о числе выравниваний.").
\end{solution}

\hrulefill

\begin{task}
	Чему равна сумма всех чисел в 9-й строке треугольника Паскаля?
\end{task}

\textbf{Ответ:} $ 512 $

\begin{solution}
	Искомая сумма равна $ C_9^0 + C_9^1+\ldots+C_9^9 = 2^9 =512 $ (см. раздел 4 "Комбинаторные тождества" задача "Сумма биномиальных коэффициентов").
\end{solution}



\end{document}